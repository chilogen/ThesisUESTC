

\lstset{
 columns=fixed,       
 numbers=left,                                        % 在左侧显示行号
 numberstyle=\tiny\color{gray},                       % 设定行号格式
 frame=none,                                          % 不显示背景边框
 backgroundcolor=\color[RGB]{245,245,244},            % 设定背景颜色
 keywordstyle=\color[RGB]{40,40,255},                 % 设定关键字颜色
 numberstyle=\footnotesize\color{darkgray},           
 commentstyle=\it\color[RGB]{0,96,96},                % 设置代码注释的格式
 stringstyle=\rmfamily\slshape\color[RGB]{128,0,0},   % 设置字符串格式
 showstringspaces=false,                              % 不显示字符串中的空格
 language=c,                                        % 设置语言
}


\chapter{固定控制类NBIOT应用的设计与实现}
基于NB-IOT的终端应用大致分为4类,分别是固定上报类,固定控制类,移动上报类和移动控制类。不同类别的应用因为数据的实时性、数据量、部署环境等的不同,对网络以及电源的需求也不同。例如对于固定控制类,
由于设备部署位置固定,常有外部电源支持,需要较强的实时性,所以对功耗需求不高,需要模块时刻保持在线状态。接下来将以BC35G模块为基础,结合stm32开发板,实现一个固定控制类物联网终端实例,演示BC35G
模块通过开发板控制通信操作,物联网应用开发以及CoAP通信过程。

\section{总体设计}

\subsection{设计目标}
使用stm32开发板,BC35G模块以及华为物联网平台,设计一个物联网演示应用。通过CoAP接入华为云平台协议接入华为云平台,能够通过华为云平台给开发板下发开灯、关灯,响铃操作,并且开发板定时上报自身LED
和无源蜂鸣器状态给华为云平台。

\subsection{设计任务}

对于一个物联网应用,其基本架构应如图\ref{物联网应用总体架构}。主要由终端设备、物联网平台和业务应用组成。终端设备作为物联网的感知层,是整个应用的核心资源,物联网平台用于管理终端设备的注册、固件
更新等,同时屏蔽复杂的设备接口,处理业务应用与终端设备的通信。业务应用则通过物联网平台暴露出的restful接口,面向用户以及管理员,为用户提供最终的服务和管理员对终端设备的管理。

\begin{figure}[h]
    \centering
  	\includegraphics[width=10cm]{物联网应用总体架构.png}
	\caption{物联网应用总体架构}
	\label{物联网应用总体架构}
\end{figure}

本次设计任务中,终端设备将通过CoAP协议与华为云平台进行二进制数据交互,并由华为云平台暴露出restful接口,所以主要工作集中与华为云平台和终端设备的开发。对于华为云平台,
需要定义设备的profile文件,以标识设备特征以及能力,也需要开发编解码插件,用于json和二进制数据的转换;对于终端设备,同样首先需要实现与华为物联网平台一致的编解码插件逻辑,从而能与物联网平台交互,
然后需要开发控制stm32资源的能力,从而使开发板能
根据华为物联网平台的相应指令做出对应动作。

\subsection{详细设计}

BC35G模块将通过一组串口与stm32开发板通信,需要使用stm32开发板上的一组串口,作为接受华为云平台控制消息和stm32开发板上传自身资源状态的通告,所以BC35G模块与stm32开发板需以图\ref{硬件连接}的方式连接。
\begin{figure}[h]
    \centering
  	\includegraphics[width=9cm]{硬件连接.png}
	\caption{硬件连接}
	\label{硬件连接}
\end{figure}

如图\ref{硬件连接}所示,stm32开发板上主要使用的控制资源是一对发光二级管和一个无源蜂鸣器,所以华为物联网平台与设备终端之间交互的消息应该有三类,分别用于终端设备上报资源状态(resource\_info),
物联网平台下发的资源控制命令(set\_resource)和物联网平台下发的资源查询命令(query\_resource)。

resource\_info需要包含3个系统资源的状态,定义为led0,led1和beep,使用0/1代表关闭、开启状态,所以使用uint8编码;
set\_resouce用于设置资源状态,所以使用num代表资源编号,分别对用 led0(0),led1(1),beep(2),state代表需要将资源设置成的状态;
query\_resouce类似set\_resource,使用num代表需要查询状态的资源编号;
三类消息中的messageId用于编解码插件识别消息类型并进行处理。
详细的消息定义如表\ref{消息模板}:

\begin{table}[h]
\caption{消息模板}
\begin{tabular}{|c|c|c|c|c|c|}
\toprule
消息类型 & 字段名称 & 数据类型 & 偏移量 & 字段解释 & 消息解释 \\
\hline
\multirow{4}{*}{resource\_info} & messageId & uint8 & 0-1 & 消息类型编号:2 & \multirow{4}{*}{模块上报消息} \\
\cmidrule{2-5}
& led0 & uint8 & 1-2 & led0 状态 & \\
\cmidrule{2-5}
& led1 & uint8 & 2-3 & led1 状态 & \\
\cmidrule{2-5}
& beep & uint8 & 3-4 & beep 状态 & \\
\hline
\multirow{3}{*}{set\_resource} & messageId & uint8 & 0-1 & 消息类型编号:0 & \multirow{3}{*}{模块控制消息} \\
\cmidrule{2-5}
& num & uint8 & 1-2 & 需要控制的资源编号 & \\
\cmidrule{2-5}
& state & uint8 & 2-3 & 资源状态 & \\
\hline
\multirow{3}{*}{query\_resource} & messageId & uint8 & 0-1 & 消息类型编号:1 & \multirow{3}{*}{触发模块上报} \\
\cmidrule{2-5}
& num & uint8 & 1-2 & 需要上报的资源编号 & \\
\bottomrule
\end{tabular}
\label{消息模板}
\end{table}

\section{华为云平台应用开发}
华为 OceanConnect物联网平台作为一个连接业务应用和物联网设备的中间层,提供了海量设备接入管理,屏蔽复杂的设备接口,支持多网络、多协议
的终端设备接入,配合华为云其他产品同时使用,可以快速构筑物联网应用。
%\begin{figure}[h]
%    \floatcontinue
%	\includegraphics[width=20cm]{oc流程.png}
%	\caption{oc流程}
%	\label{oc流程}
%\end{figure}

\subsection{创建项目}
项目作为华为物联网平台上的一个应用空间,用于区分不同应用场景下的调试开发,每个项目有一个应用ID作为项目的唯一标识,华为物联网平台用于区
分通过Restful接口到来的应用服务器请求以路由到对应的项目空间。

在华为OceanConnect开发中心上创建项目如图\ref{创建项目1},获取平台分配的应用ID和应用秘钥,用于后期应用服务器连接华为物联网平台。
\begin{figure}[H]
    \centering
	\includegraphics[width=8cm]{创建项目1.png}
	\caption{创建项目1}
	\label{创建项目1}
\end{figure}
\begin{figure}[H]
    \centering
	\includegraphics[width=8cm]{创建项目3.png}
	\caption{创建项目3}
	\label{创建项目3}
\end{figure}
\begin{figure}[H]
    \centering
	\includegraphics[width=8cm]{创建项目2.png}
	\caption{创建项目2}
	\label{创建项目2}
\end{figure}


\subsection{定义产品}
产品是指一类具备相同能力和特征的设备,一个产品包含产品模型、编解码插件等资源。应用层协议选用CoAP协议。由于使用JSON的数据格式对能耗消耗太大,
不适用于物联网设备,所以选用二进制码流的数据格式,通过开发编解码插件解析。定义产品如图\ref{oc产品定义}。
\begin{figure}[H]
    \centering
	\includegraphics[width=8cm]{oc产品.png}
	\caption{oc产品定义}
	\label{oc产品定义}
\end{figure}


\subsection{定义profile与编解码插件}
profile是描述产品设备信息的文件,定义了设备与应用服务器交互的字段及格式。其主要包含产品信息、服务能力以及维护能力。

实验设备资源主要使用了两个发光二级管以及一个无源蜂鸣器,支持两种操作,分别是触发设备上报设备资源状态和设置设备资源状态,因此设备具有三种属性led0、led1和beep,支持两条下发命令
set\_resource和query\_resource。该设备可用如表\ref{profile格式}的profile文件描述。
\begin{table}[h]
\caption{profile格式}
\begin{tabular}{|c|c|c|c|c|}
\toprule
\multicolumn{5}{|l|}{属性列表} \\
\hline
属性名称 & 类型 & 取值 & \multicolumn{2}{|c|}{描述}  \\
\hline
led0 & int & 0~1 & \multicolumn{2}{|l|}{0:led0关闭 1:led0开启} \\
\hline
led1 & int & 0~1 & \multicolumn{2}{|l|}{0:led1关闭 1:led1开启} \\
\hline
beep & int & 0~1 & \multicolumn{2}{|l|}{0:蜂鸣器关闭 1:蜂鸣器开启} \\
\toprule
\multicolumn{5}{|l|}{命令列表} \\
\hline
命令名称 & 字段属性 & 字段名 & 取值 & 描述 \\
\hline
\multirow{2}{*}{set\_resouce} & 请求 & num & 1-3 & 资源编号 \\
\cmidrule{2-5}
&请求&state&0-1&资源状态 \\
\hline
query\_resouce & 请求 & num & 1-3 & 资源编号 \\
\hline
\bottomrule
\end{tabular}
\label{profile格式}
\end{table}

二进制数据格式需要编解码插件才能解析,在oc平台上设置好profile 文件后,通过将相应属性值与消息模板\ref{消息模板}一一对应,oc平台将会为自身自动生成编解码器,
同时也需要在设备端实现一致的编解码逻辑。


\subsection{接入设备}
华为物联网平台调测可以使用虚拟设备和现实物理设备,当设备侧未开发完成时可以只使用虚拟设备进行调测。对于真实物理设备,华为物联网平台需要设备的唯一标识来认证设备。
此处我选择使用IMEI号。使用PC通过串口连接设备,使用AT+CGSN=1命令查询设备IMEI号,正常情况下设备返回 +CGSN:<IMEI> OK。在oc平台上新增真实设备,填入设备名称和IMEI号,
华为物联网平台将能够识别该设备。
\begin{figure}[h]
	\includegraphics[width=15cm]{oc接入方式.png}
	\caption{oc接入方式}
	\label{oc接入方式}
\end{figure}


\section{stm32终端开发}
stm32终端选择一块搭载stm32f103zet的开发板,板载4组串口,第一组可用于USB通信。本次设计使用到其中第二组串口,对应引脚为PA2和PA3。使用无源蜂鸣器及一组发光二极管,作为设备终端的资源,
对应引脚为PB8,PB5,PE5。使用key0-3,作为外部中断输入,对应模块的初始化、重连和退网,对应引脚为PE2,PE3,PE4。

\subsection{前后台程序}
为了管理运行在MCU上的不同任务,比如数据的发送接受,资源的操作,可以使用成熟的操作系统比如usos管理多任务,也可以使用前后台加中断的方式管理。

在前后台系统的设计中,主应用程序是一个无限循环,可以看做是后台程序,掌管整个系统资源的分配管理和任务调度。为了达成较高的实时性系统要求,前台程序一般只在中断服务程序中标记时间的发生,不做其它多余的操作。
同时具体到stm32嵌入式开发使用的系统HAL库,因为系统时钟中断使用的最低优先级,如果阻塞在前台程序的中断服务函数中,会造成系统时钟失效,甚至永久阻塞。

以模块的入网初始化、入网重连和模块关机事件为例,通过外部中断的方式接受按键输入,在中断服务程序中标记入网初始化、入网重连和模块关机事件的发生,在后台中调用相应的事件处理函数执行,代码如下:

\begin{lstlisting}

void HAL_GPIO_EXTI_Callback(uint16_t GPIO_Pin) {
    if(!initComplete){
        return;
    }

    switch (GPIO_Pin) {
        case GPIO_PIN_2: {
            moduleInit = 1;
            break;
        }
        case GPIO_PIN_3: {
            moduleReconnect = 1;
            break;
        }
        case GPIO_PIN_4: {
            moduleClose = 1;
            break;
        }
        default: {
        }
    }
}

void main()}{
  while (1) {
        if (moduleInit) {
            NBInit();
            moduleInit = 0;
        }
        if (moduleClose) {
            NBClose();
            moduleClose = 0;
        }
  }
}
\end{lstlisting}

然而实际上,前后台系统的实时性并不能达到理想状态,这是由于没有为不同任务分配优先级,一般采用FIFO的方式调度。但因为按键触发初始化入网、入网重试、接收消息、发送消息等任务实际上并不会有极高的使用概率和实时性要求,接受和发送
消息任务都是读取写入BC35G模块的flash,所以使用前后台程序能够满足需求,并降低复杂性。

\subsection{指令单元序列}
由于通过串口控制BC35G模块运行既是通过发送一系列AT指令,这些指令序列有重复性和相似性,所以将指令视为单元,通过单元的串联组合完成一个设备功能,比如初始化入网、入网重试、接收消息、发送消息。
以模组关机为例,需要执行关闭网络(\_\_NB\_NETCLOSE)和关闭射频单元(\_\_NB\_CloseCFun)操作,两个操作组合形成功能NBClose,如此能极大方便新功能的增加和旧功能的维护,代码如下:

\begin{lstlisting}
uint8_t (*CloseProc[])()={
        __NB_NETCLOSE,
        __NB_CloseCFun,
};

int8_t __NB_NETCLOSE(){
    char resBuff[UART_BUFFER_SIZE];
    memset(resBuff,0,sizeof resBuff);
    uint8_t res_size;
    char com[30];
    memset(com, 0, sizeof com);
    sprintf(com, NB_CGATT, __NB_ZERO);
    NBCommand((uint8_t*)com, strlen(com), resBuff, &res_size);

    char *strx=NULL;
    strx = strstr((const char *) resBuff, __NB_OK);

    if (strx == NULL) {
        return 0;
    }
    return 1;
}

uint8_t __NB_CloseCFun() {
    char resBuff[UART_BUFFER_SIZE];
    memset(resBuff,0,sizeof resBuff);
    char com[30];
    memset(com, 0, sizeof com);
    sprintf(com, NB_ATCFUN, __NB_ZERO);
    uint8_t res_size;
    NBCommand((uint8_t *)com, strlen(com), resBuff, &res_size);

    char *strx = NULL;
    strx = strstr((const char *) resBuff, NB_OK);

    if (strx == NULL) {
        return 0;
    }
    return 1;
}

uint8_t NBClose(){
    unsigned int it = 0, procNum=sizeof CloseProc;
    procNum=procNum/4;
    for (it = 0; it < procNum; it++) {
        if (!CloseProc[it]()) {
            NBERROR(it);
            it--;
        }
    }
    return 1;
}
\end{lstlisting}

\subsection{串口DMA通信}

stm32开发板控制bc35g模块是通过串口的方式,bc35g串口比特率为9600。为了接收变长的串口数据,可以有以下几种方式:

BC35G模块串口传输数据以'lr cr'为分隔符,以软件的方式,设置超时接收固定长度并以'lr cr'分割,放入缓冲区中。由于模块涉及网络操作等原因,不同AT指令的响应时间有很大差距,所以超时时
间不好确定,同时由于MCU等待模块输入,对效率影响较大。

利用定时器中断方式可以解决超时时间设置的问题。一个字节的数据有 起始位+数据+结束位共10位,在模块串口比特率为9600的情况下,传输一个字节需要104us。同时由于两组数据之间需要间隔3.5字
符,可以设置定时器中断为5ms。在串口接收中断服务函数中开启定时器中断,每接收一个字符则重置定时器,当定时器超时时可以认为一组数据接收完毕,在定时器中断函数中将接收到的数据放入缓存中。
但是由于是一字节一字节接收,而且MCU仍然参与接收过程,所以效率仍有提升必要。

为了进一步提升效率,可以使用DMA方式接收数据。为了区分两组数据,开启总线空闲中断,当DMA传输完毕时触发总线空闲中断,在总线空闲中断中标记数据就绪。DMA方式能获得更好的效率。

\begin{lstlisting}
/* uart.c
* 检测空闲中断,标记缓存数据可用
*/

uint8_t UART2DATAREADY = 0;
uint8_t UART2RXBUFFER[UART_BUFFER_SIZE];

void USER_UART_IRQHandler(UART_HandleTypeDef *huart) {
    if (huart == &huart2) {
        if (RESET != __HAL_UART_GET_FLAG(&huart2, 
                UART_FLAG_IDLE)) {
            __HAL_UART_CLEAR_IDLEFLAG(&huart2);
            HAL_UART_DMAStop(&huart2);
            UART2DATAREADY = UART_BUFFER_SIZE - 
            __HAL_DMA_GET_COUNTER(&hdma_usart2_rx);
        }
    }
}

/* nbiot.c
* 封装AT指令执行模块,接收模块串口回传数据
*/
func NBCommand(AT_command) {
    UART_Transmit(AT_command);
    HAL_UART_Receive_DMA(&huart2, 
                        UART2RXBUFFER, 
                        UART_BUFFER_SIZE);
    Delay(100); //降低UART2DATAREADY datarace概率
\end{lstlisting}    

\subsection{初始化及入网}
BC35G模块初始化需要首先使用AT+NCDP=<ip,port>命令设置云平台CoAP协议接入地址,该地址可以从华为物联网平台的项目对接信息中获取。

根据USIM卡对应的不同运营商和表\ref{运营商band分配},使用AT+NBAND=<band>命令设置设备运行频段。

同时为了达到固定控制类应用对时延的要求,需要使用AT+CEDRXS=0,5和AT+CPMS=0关闭edrx和省电模式PSM。

模块默认开启数据自动上报,当接收到物联网平台消息是就会自动通过串口上报。由于为了避免与stm32的控制消息在串口结合在一起后期分离不变,所以使用AT+NNMI关闭串口数据自动上
报,根据NBIOT的drx模式,定时按需使用AT+NQMGR和AT+NMGR从缓存中获取消息。
最终触发入网并查询入网状态,需要执行的流程及AT指令如图\ref{入网流程}

\begin{figure}[H]
    \centering
	\includegraphics[width=13cm]{入网流程.png}
	\caption{入网流程}
	\label{入网流程}
\end{figure}


如果入网失败,可以尝试重连,重练操作序列如图\ref{入网重试}

\begin{figure}[H]
    \centering
	\includegraphics[width=13cm]{入网重试.png}
	\caption{入网重试}
	\label{入网重试}
\end{figure}

当设备不处于第一次开机的流程,入网操作只需要打开射频功能以及触发入网,如图

\subsection{模组通信}

bc35g模块涉及消息发送与接收的有如下四个命令:
\begin{table}[h]
\caption{bc35g模块收发命令}
\begin{tabular}{|c|c|}
\toprule
命令 & 描述 \\
AT+NMGS=<length>,<data> & \makecell[l]{data为16进制数据,length为data长度,\\用于向IOT平台发送数据} \\
AT+NIMI=0/1 & \makecell[l]{关闭/开启自动上报,模块接收到消息是会\\自动发往串口} \\
AT+NQMGR & \makecell[l]{查询自开机以来接收到的消息状态} \\
AT+NMGR & \makecell[l]{读取缓存中最早一条未被未被处理的消息} \\
\bottomrule
\end{tabular}
\label{bc35g模块收发ta}
\end{table}

如果开启自动上报,则服务器下发内容有可能与串口控制消息重叠,需要消耗额外资源去提取。为了节约系统消耗,所以关闭自动上报,简单定时轮询缓存中是否有接收到新消息。

\begin{lstlisting}

//nbiot.c
func readMsg(){
    received=NBCommand("AT+NQMGR")
    for read to received{
        data+=NBCommand("AT+NMGR")
    }
    data+="command_end"
    read=received;
    return data;
}

//main.c
while(1){
    Delay(1000);
    data=readMsg();
    process(data);
}
\end{lstlisting}

\subsection{编解码插件}
由于华为物联网平台暴露给应用服务器的Restful接口使用的是json格式数据,需要使用编解码插件将设备终端的二进制数据转换为json数据。因此,也需要在设备终端上将自定义结构数据编码为华为物联网平台编解码插件输入格式相同的二进制数据。
根据在华为物联网平台上定义的消息模板\ref{消息模板},该二进制数据格式为: payload = hex(<length><messageid><data>),length和messageid都为uint8类型,所以各占两个char,data 按照偏移量排列,初始值为‘00’,因此定义三种消息结构体分别对应
模块上报消息(HWOCReportMSG)、模块控制消息(HWOCSetMSG)和模块查询消息(HWOCQueryMSG),同时还有处理中间消息(HWOCMSG)如下:

\begin{lstlisting}
typedef struct {
    uint8_t msgId;
    uint8_t led0;
    uint8_t led1;
    uint8_t beep;
}HWOCReportMSG;

typedef struct {
    uint8_t msgId;
    uint8_t num;
    uint8_t state;
}HWOCSetMSG;

typedef struct {
    uint8_t msgID;
    uint8_t num;
}HWOCQueryMSG;

typedef struct  {
    uint8_t msgId;
    uint8_t *data;
}HWOCMSG;
\end{lstlisting}



\subsection{退网关机}

当设备关机时需要释放与运营商的连接,须执行以下序列\ref{模块关机}
\begin{figure}[H]
    \centering
	\includegraphics[width=7cm]{模块关机.png}
	\caption{模块关机}
	\label{模块关机}
\end{figure}

\section{验证过程以及结论}

\subsection{PC串口通信验证}
BC35G模块通过NBC连接PC,为了使操作系统能够识别USB转TTL转接模块,需要安装CH341芯片驱动。CH341是沁恒公司生产的一款USB总线转接芯片,实现了通过USB总线提供异步串口、并口、打印口和常用的同步串行接口,在其
官网上提供驱动源码下载,编译后以内核模块的方式加载到操作系统中,安装步骤如下:
\begin{enumerate}
\item 在沁恒公司官网(www.wch.cn/download/CH341SER\_LINUX\_ZIP.html)下载CH341ESR\_LINUX.ZIP并解压
\item 进入源码目录执行编译 make load 生成ch341.ko内核模块
\item 使用insmod命令加载ch341.ko模块
\item 使用dmesg查看内核日志输出,在已插入USB转TTL转接模块的情况下,内核日志中显示识别设备为ttyUSB0如图\ref{ch341驱动加载},在 /dev 设备中同样可以看见转接设备如图\ref{devusb0}
\end{enumerate}

\begin{figure}[H]
    \centering
	\includegraphics[width=10cm]{ch341驱动加载.png}
	\caption{ch341驱动加载}
	\label{ch341驱动加载}
\end{figure}

\begin{figure}[H]
    \centering
	\includegraphics[width=10cm]{devusb0.png}
	\caption{devusb0}
	\label{devusb0}
\end{figure}

minicom是linux下的一款串口通信软件,类似windows的超级终端,可用于调试交换机等串口设备,源中已经包含了minicom软件包,使用apt install minicoom即可安装。
安装完毕后,使用 minicom -s 配置串口为 /dev/ttyUSB0,并设置比特率为9600bps,即可与bc35g模块通信。

按顺序执行以表\ref{PCAT指令实验}AT指令,即可完成向华为物联网平台发送消息的功能,执行效果如图\ref{PC命令行实验}

\begin{table}[h]
\caption{PCAT指令实验}
\begin{tabular}{|l|l|}
\toprule
命令 & 描述 \\
AT & \makecell[l]{查询模块状态,模块就绪返回 OK} \\
AT+NCONFIG=AUTOCONNECT,true & \makecell[l]{开启手动配置入网} \\
AT+NCDP=49.4.85.232,5683 & \makecell[l]{设置华为物联网平台CoAP接入地址} \\
AT+NRB & \makecell[l]{重启设备} \\
AT+NBAND=5 & \makecell[l]{设置运行频段为移动频段band5} \\
AT+NNMI=1& \makecell[l]{开启模块自动上报,模块接收到消息是会自动发往串口} \\
AT+CFUN=1 & \makecell[l]{打开射频功能,准备入网} \\
AT+CGATT=1 & \makecell[l]{触发开始入网} \\
AT+CGATT? & \makecell[l]{查询入网状态,入网成功返回1} \\
\bottomrule
\end{tabular}
\label{PCAT指令实验}
\end{table}

\begin{figure}[H]
    \centering
	\includegraphics[width=10cm]{PC命令行实验.png}
	\caption{PC命令行实验}
	\label{PC命令行实验}
\end{figure}

\subsection{开发板实验}

编译代码生成二进制固件后,需要将其烧写进开发版内存。openocd是一个由Dominic Rath创建的嵌入式调试工具,需要配合调试适配器,比如JTAG,SWD等。其同样可以用于固件下载更新。
由于MCU是stm32f103zet,使用的调试适配器为stlink v2,需要编写openocd配置文件如下,以选择配置stlink v2 调试适配器和stm32f1x系列起始地址等内容。
\begin{lstlisting}
# config
source [find interface/stlink-v2.cfg]
transport select hla_swd
source [find target/stm32f1x.cfg]
reset_config srst_only
\end{lstlisting}
然后使用命令 openocd -f config -c <binary>.elf 将固件下载到开发板上。

系统初始化时LED处于关闭状态,华为物联网平台显示设备离线,两者状态如图\ref{设备离线端}、图\ref{设备离线}

\begin{figure}[H]
    \centering
	\includegraphics[width=7cm]{设备离线端.png}
	\caption{设备离线端}
	\label{设备离线端}
\end{figure}


\begin{figure}[H]
    \centering
	\includegraphics[width=7cm]{设备离线.png}
	\caption{设备离线}
	\label{设备离线}
\end{figure}

入网需要大概40s到90s,入网成功后,红色LED0亮起,在设备终端表示入网成功,设备初始化完成,同时华为物联网平台也显示设备上线,可以进行数据收发和控制操作。
此时终端设备的状态如图\ref{设备上线端},华为物联网平台状态如图\ref{设备上线}
\begin{figure}[H]
    \centering
	\includegraphics[width=7cm]{设备离线端.png}
	\caption{设备上线端}
	\label{设备上线端}
\end{figure}


\begin{figure}[H]
    \centering
	\includegraphics[width=7cm]{设备离线.png}
	\caption{设备上线}
	\label{设备上线}
\end{figure}

通过下发控制命令开启、关闭LED灯和触发设备信息上报,设备端和云端状态如下:
\begin{figure}[H]
    \centering
	\includegraphics[width=7cm]{发送开灯.png}
	\caption{发送开灯}
	\label{发送开灯}
\end{figure}


\begin{figure}[H]
    \centering
	\includegraphics[width=7cm]{发送开灯端.png}
	\caption{发送开灯端}
	\label{发送开灯端}

\end{figure}\begin{figure}[H]
    \centering
	\includegraphics[width=7cm]{设备上报.png}
	\caption{设备上报}
	\label{设备上报}
\end{figure}

\section{本章小结}

本章通过简单模拟一个固定控制类应用,在STM32开发板上实现对BC35G模块的通信控制,通过CoAP协议,完成查询、上报、控制资源状态三种消息类型的传输,验证了BC35G模块的功能