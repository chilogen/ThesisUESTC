\chapter{全文总结与展望}

\section{全文总结与结论}

本文详细介绍了NBIOT通信协议以及模块外围电路设计相关内容,并通过相关验证使用stm32开发板验证了bc35g模块的通信操作以及使用,相关代码可在github\cite{wsq_git}获取

\section{后续工作展望}

模块的实际设计与开发需要投入不少资源用于生产试错的过程,就如FPGA的出现为芯片设计提供了极大地便利一样,SDR(Software Defined Radio)通过软件来模拟传统的硬连线方式实现无线电通信,只需使用不同的软件就能在通用PC上实现一个通信模块具有的功能,不仅方便了无线电爱好者低成本的探索无线电世界,对于通信协议的研究也提供了快捷的方式。预期通过对通信原理的深入学习,可以实现用于gnuradio的NB-IOT插件,更加方便对NB-IOT技术的学习。