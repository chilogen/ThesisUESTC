\chapter{全文总结与展望}

\section{全文总结}
(temp-TODO:代码地址 https://github.com/chilogen/stm32\_test/tree/feature/nbiot-bs)

(temp-TODO:论文地址 https://github.com/chilogen/ThesisUESTC)


\section{后续工作展望}

模块的实际设计与开发需要投入不少资源用于生产试错的过程,就如FPGA的出现为芯片设计提供了极大地便利一样,SDR(Software Defined Radio)通过软件来模拟传统的硬连线方式实现无线电通信,只需使用不同的软件就能在通用PC上实现一个通信模块具有的功能,不仅方便了无线电爱好者低成本的探索无线电世界,对于通信协议的研究也提供了快捷的方式。预期通过对通信原理的深入学习,可以实现用于gnuradio的NB-IOT插件,更加方便对NB-IOT技术的学习。