\thesischapterexordium

\section{研究工作的背景与意义}

物联网的概念自1999年提出以来就备受关注,随着物联网在各行业的需求逐渐被挖掘,无量网技术的研究和应用都得到了极大的发展。


低功耗广域网(Low-Power Wide-Area Network, LPWNA)作为无物联网一个重要的应用分支,具有覆盖能力强,能耗低的特点。LoRa、Sigfox、NB-IoT等技术相继涌现,其中,NB-IOT由于其使用授权频谱等优势,收获了众多学者的关注和产生了许多行业应用,在5G标准中也占有一席之地。

2015年9月,NB-CIoT(Narrow Band Celluar IoT)与爱立信的NB-LTE方案融合,形成了NB-IOT方案。华为、高通、爱立信的那个在同年12月合作提出的NB-IoTf方案,在2016年6月获得3GPP批准,并在2019年7月获得ITU确认,成为解决5G mMTC(massive Machine Type Communication)场景下的技术标准。

在协议演进的过程中,研究人员通过对物理层的研究,对NB-IOT的功耗、传输时延、与LTE系统之间的干扰进行了分析与建模。比如 Migabo E等人的论文(TODO:引用)设计了NB-IOT下行链路的数学模型,通过理论分析和模拟了NB-IOT的预期能耗,同时讨论了数据传输速率以及网络传输时延的问题;Kim H等的论文(TODO:引用)对NB-IOT与线性通信系统的干扰进行了分析,提出在NB-IOT频带两侧设置保护带宽的建议。同时在网络构建与多技术融合方面也有很多成果。

众多模块和芯片厂商也对NB-IOT技术的落地做了充足的准备。华为在收购Neul的基础上在2016年就展示了承载NB-IOT的Boudica芯片,上海移远通信技术有限公司也通过搭载该系列芯片开发出了BC95系列模组。

运营商也根据工信部2020年NB-IOT网络实现全国普遍覆盖的要求,积极部署NB-IOT基站,目前中国电信拥有40余万个基站(TODO:引用),联通和移动也紧随其后。

计算电磁学方法\citing{wang1999sanwei, liuxf2006, zhu1973wulixue, chen2001hao, gu2012lao, feng997he}从时、频域角度划分可以分为频域方法与时域方法两大类。频域方法的研究开展较早,目前应用广泛的包括:矩量法(MOM)\citing{xiao2012yi,zhong1994zhong}及其快速算法多层快速多极子(MLFMA)\citing{clerc2010discrete}方法、有限元(FEM)\citing{wang1999sanwei,zhu1973wulixue}方法、自适应积分(AIM)\citing{gu2012lao}方法等,这些方法是目前计算电磁学商用软件
\footnote{脚注序号“\ding{172},……,\ding{180}”的字体是“正文”,不是“上标”,序号与脚注内容文字之间空1个半角字符,脚注的段落格式为:单倍行距,段前空0磅,段后空0磅,悬挂缩进1.5字符;中文用宋体,字号为小五号,英文和数字用Times New Roman字体,字号为9磅;中英文混排时,所有标点符号(例如逗号“,”、括号“()”等)一律使用中文输入状态下的标点符号,但小数点采用英文状态下的样式“.”。}
(例如:FEKO、Ansys 等)的核心算法。由文献\cite{feng997he,clerc2010discrete,xiao2012yi}可知

\section{NB-IOT技术国内外研究历史与现状}
时域积分方程方法的研究始于上世纪60 年代,C.L.Bennet 等学者针对导体目
标的瞬态电磁散射问题提出了求解时域积分方程的时间步进(marching-on in-time,
MOT)算法。

\section{本文的主要贡献与创新}
本文将于上海移远通信有限公司的BC35G模块为例,阐述NB-IOT技术及相关模块设计原理

\section{本论文的结构安排}
本文的章节结构安排如下:
