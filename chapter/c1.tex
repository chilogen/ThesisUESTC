\thesischapterexordium

\section{研究工作的背景与意义}

物联网的概念自1999年提出以来就备受关注,随着物联网在各行业的需求逐渐被挖掘,物联网技术的研究和应用都得到了极大的发展。


低功耗广域网(Low-Power Wide-Area Network, LPWNA)作为无物联网一个重要的应用分支,具有覆盖能力强,能耗低的特点。LoRa、Sigfox、NB-IoT等技术相继涌现,其中,NB-IOT由于其使用授权频谱等优势,收获了众多学者的关注和产生了许多行业应用,在5G标准中也占有一席之地。

2015年9月,NB-CIoT(Narrow Band Celluar IoT)与爱立信的NB-LTE方案融合,形成了NB-IOT方案。华为、高通、爱立信的那个在同年12月合作提出的NB-IoTf方案,在2016年6月获得3GPP批准,并在2019年7月获得ITU确认,成为解决5G mMTC(massive Machine Type Communication)场景下的技术标准。

在协议演进的过程中,研究人员通过对物理层的研究,对NB-IOT的功耗、传输时延、与LTE系统之间的干扰进行了分析与建模。比如 Migabo E等人的论文(TODO:引用)设计了NB-IOT下行链路的数学模型,通过理论分析和模拟了NB-IOT的预期能耗,同时讨论了数据传输速率以及网络传输时延的问题;Kim H等的论文(TODO:引用)对NB-IOT与线性通信系统的干扰进行了分析,提出在NB-IOT频带两侧设置保护带宽的建议。同时在网络构建与多技术融合方面也有很多成果。

众多模块和芯片厂商也对NB-IOT技术的落地做了充足的准备。华为在收购Neul的基础上在2016年就展示了承载NB-IOT的Boudica芯片,上海移远通信技术有限公司也通过搭载该系列芯片开发出了BC95系列模组。

运营商也根据工信部2020年NB-IOT网络实现全国普遍覆盖的要求,积极部署NB-IOT基站,目前中国电信拥有40余万个基站(TODO:引用),联通和移动也紧随其后。

\section{本论文的结构安排}

论文总共分五章,结构安排如下:

第一章绪论部分主要阐述NBIOT技术的产生与发展,内容包括几大通信技术公司联合提出协议规范,列举一些学者在物理层对NBIOT协议能耗、时延的研究,以及NBIOT在5G标准中扮演的角色。最终结合产业政策推动等产业因素,得出NBIOT技术将在未来得到蓬勃发展,导致集成模块的需求也会增加

第二章主要说明NBIOT设计的技术及其通信规范,包括控制模块需要的AT指令集,NBIOT获得分配的通信频段,部署方式,工作方式以及对两个用于NBIOT的应用层协议(CoAP和MQTT)进行对比,从理论知识上理解NBIOT技术的能耗、时延等特点。

第三章从模块设计出发,分析了bc35g模块的供电管理模块、射频前端模块、串口通信模块、USIM卡座模块等的设计原则和电路分析。

第四章则是实验与验证部分,通过华为云平台、NBIOT模块、stm32开发板,验证NBIOT通信过程及探讨不同的NBIOT应用设计要求。

第五章总结与展望未来工作。总结NBIOT通信规范以及模块设计,展望通过软件无线电技术研究NBIOT的工作。