\chapter{NB-IOT通信技术基础}


\section{AT指令}
AT指令集用于从终端设备或数据终端设备通过终端适配器向控制移动台发送控制指令,3GPP TS 27.007 规定了用于控制GSM手机或者modem的AT命令,比如 AT+CIMI 用于获取设备的IMEI(Intemational Mobile Subscriber Identity),AT+CSQ 用于获取信号强度;还管理GSM和SMS短信功能的指令,例如AT+CNGC用于发送SMS短信;另有模块通用指令比如 AT+NRB用于重启模块;还有用于特定IOT平台的命令,比如AT+NCDP用于查询华为IOT平台CDP服务器的IP/端口。
AT指令分为测试、读取、写、执行四类指令,形式如表\ref{AT指令形式}

\begin{table}[h!]
\caption{AT指令形式}
\begin{tabular}{lll}
\toprule
类型&形式&解释\\
\midrule
测试指令&AT+<cmd>=?&获取参数可能的值范围\\
读取指令&AT+<cmd>?&读取参数值\\
写指令&AT+<cmd>=p1,[p2,[p3[......]]]&设置参数值\\
执行指令&AT+<cmd>&执行命令\\
\bottomrule
\end{tabular}
\label{AT指令形式}
\end{table}


\section{NB-IOT模块的工作状态}
NB-IOT省电的特点,很大程度得益于3种工作模式的切换,通过牺牲一定的实时性,停止一部分功能,从而达到省电目的。NB-IOT三种工作模式的切换如下图所示(补充图片)

当NB-IOT模块注册入网后,处于connected态,此时模块可以收发数据,当无上行数据后,启动不活动定时器,到期后进入IDLE态,开始TAU计时,关闭发射,保留接收功能,可对下行数据进行处理,如有下行激活指令,则切换到connected态,否则active timeer到期后进入PSM态,关闭收发功能,仅保留必要的时钟。TAU到期后切换到connected态

对于不同场景下的的NB-IOT应用,通过选取合适的定时器时间及工作状态切换流程,能满足不同的实时性以及能耗的要求。比如固定的控制类物联网应用,由于有可靠的电源保障,对实时性的需求大于低能耗的需求,可以使模块一直保持在Active模式。
\section{应用层协议}

物联网设备多为系统资源受限设备,对网络的要求也与普通互联网应用有很大不同。目前流行的为物联网受限设备设计的应用层协议主要是CoAP、MQTT-SN、XMPP、AMQP等(TODO:引用)。接下来对应用较为广泛的CoAP和MQTT协议进行对比。

\subsection{CoAP}
CoAP与http一样,基于REST架构的应用层协议,但传输层使用UDP替换,自身则是一个双层结构(消息层和请求响应层),并在消息层实现了多播、拥塞控制、消息可靠性保证等机制。作为一个轻量级协议,CoAP协议的最小报文仅为4字节.

\begin{figure}[h]
	\includegraphics[width=14cm]{CoAP报文.png}
	\caption{CoAP的消息格式}
	\label{CoAP报文}
\end{figure}

  
\begin{table}[h!]
\caption{CoAP字段}
\begin{tabular}{ccc}
\toprule
字段名 & 比特数 & 解释\\
\midrule
Ver&2&代表协议版本,目前为 0x01 \\
Type&2&\makecell[c]{消息类型,CON为可靠传输,需要响应,响应消息类\\型为 ACK 和 RSTNON 为不可靠传输,不需要响应}\\
TKL&4&Token 的长度,目前为 0~8B \\ 
Code&8&	\\
MessageID&16&消息ID,用于标记重发消息\\
Token&&可选,长度为TKL,用于消息安全性或对应一个请求和响应\\
\bottomrule
\end{tabular}
\label{coap字段}
\end{table}


\subsection{MQTT}

MQTT是一种发布/订阅传输协议,基于TCP/IP协议栈构建,具有低开销,低带宽占用的优点,协议中有三种身份:publisher、broker和subcriber,broker起到解耦合的作用。MQTT提供三种消息传递的服务质量水平:

Qos 0 代表消息至多会传输一次,因为NBIOT底层依赖于不可靠的UDP传输,所以会发生消息的丢失。

Qos 1 代表消息至少会传输一次,但因为重发机制可能会导致消息重复。

Qos 2 代表q确保消息会到达一次,保证不会出现消息丢或重复的现象,适用于非幂等的请求传输。

%\begin{table}[h!]
%\caption{MQTT服务质量}
%\begin{tabular}{cc}
%\toprule
%值 & 解释 \\
%\midrule
%Qos 0 & 至多一次 \\
%Qos & 至少一次 \\
%Qos & 只有一次 \\
%\bottomrule
%\end{tabular}
%\label{tablea}
%\end{table}



\begin{figure}[h]
	\includegraphics[width=12cm]{MQTT报文.png}
	\caption{MQTT报文结构}
	\label{MQTT报文}
\end{figure}

\begin{table}[h!]
\caption{MQTT报文字段}
\begin{tabular}{ccl}
\toprule
字段名 & 比特数 & 解释\\
\midrule
Control Packet type	&4&控制报文类型,4位无符号数\\
Flags	&4&	\\
Remaining Length&&\makecell[c]{剩余长度表示当前报文余下的负载数据长度\\(不包含自身),使用Base 128 Varints编码}\\
\bottomrule
\end{tabular}
\label{mqtt字段}
\end{table}

\subsection{CoAP和MQTT的异同}


  MQTT协议使用发布/订阅模型,CoAP协议使用请求/响应模型;

  MQTT会通过网关在客户机之间创建连接,CoAP协议则是无连接协议;

  MQTT通过中间代理传递消息的多对多协议,CoAP协议是Server和Client之间消息传递的单对单协议;
  
  MQTT不支持带有类型或者其它帮助Clients理解的标签消息,CoAP内置内容协商和发现支持,允许设备彼此窥测以找到交换数据的方式。


\section{工作频段}
无线电频谱作为一种资源,由国家统一分配。相比于其它的LPWAN技术,NBIOT的一大优势就是它运行于授权频段,拥有较少的频段干扰,可以提供更好的服务质量。NB-IOT沿用LTEd定义的标准,能与现有的蜂窝网络基站融合快速部署,全球主流的部署频段是800MHz和900MHz,在中国,中国电信将NB-IOT部署在800MHz的频段上,而中国移动和中国联通则把NB-IOT部署在900MHz的频段上。


\begin{table}[h!]
\caption{运营商band分配}
\begin{tabular}{ccccc}
\toprule
频段&中心频率&上行频率&下行频率&\multirow{2}{*}{运营商}\\
&(MHZ)&(MHZ)&(MHZ)&\\
\midrule
Band 5&850&824-849&869-894&中国电信\\
\midrule
Band 8&900&880-915&925-960&\makecell[c]{中国移动\\中国联通}\\
\bottomrule
\end{tabular}
\label{运营商band分配}
\end{table}


\section{部署方式}
NB-IOT支持独立、带内、保护频段三种部署方式。
独立部署是指在LTE的频段外选择独立的空闲频段部署,比如部署在GSM频段,并加上保护频段,能避免与LTE信号相互干扰,但这种方式对紧张的无线电频段资源不友好;
带内部署是指选择现有LTE系统内的一个PRB(Physical Resource Block)部署,但需要考虑到与LTE系统的干扰问题(TODO:引用);
保护带部署是指利用LTE系统内尚未被利用的位于保护带的PRB,该种方式提高了无线电资源的利用率,但需要选取大于NB-IOT系统需要的200Khz带宽。
三种部署方式如图(TODO:图)


\section{物理层协议}

%https://blog.csdn.net/Simon_csx/article/details/79378337
NBIOT上行采用SC-FDMA多址方式,使用多载波和单载波两种传输方式。多载波适用于LTE相同的15khz子载波间隔,对应不同的速率场景,0.5ms时隙,1ms子帧长度,每个时隙包含7个符号

下行则是采取的OFDMA多址方式

\section{本章小结}

本章主要阐述了以下内容:

用于控制通信模块的AT指令集的四种类型指令的格式,分别为测试、读取、写以及执行指令。第四章实验部分将通过AT指令控制模块完成入网以及通信过程。

还有NB-IOT的工作状态切换以及用于物联网应用的应用层协议,了解对物联网应用开发需要注意的实时性与能耗取舍的问题。

然后通过NB-IOT系统的部署方式及物理层协议对NB-IOT通信技术有更深的理解。